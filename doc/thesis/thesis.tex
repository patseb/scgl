\documentclass[a4paper,12pt]{thesis}

\autor{Patryk Kwiatkowski}
\tytul{Podstawowa biblioteka grafów~w~C}
\tytulAng{Basic graph library in~C}
\promotor{dr inż. Ireneusz Szcześniak}
\rok{2012}
\kierunek{Informatyka}
\specjalnosc{Sieciowe Technologie Informatyczne}
\numerAlbumu{101510}
\studia{stacjonarne}
\stopien{II}

\newcommand{\eng}[1]{(\emph{#1})}

\begin{document}

\stronaTytulowa

\tableofcontents

\chapter*{Cel pracy}
\addcontentsline{toc}{chapter}{Cel pracy}
Cel i \index{lol}zakres pracy\cite{bib:test}
\chapter*{Wstęp}
\addcontentsline{toc}{chapter}{Wstęp}
\chapter{Wybrane zagadnienia teorii grafów}
\section{Podstawowe pojęcia}
\section{Algorytm Dijkstry}
\chapter{Biblioteki w systemach Unix / Linux}
\section{Statyczne}
\section{Współdzielone}
\chapter{Realizacja biblioteki}
TODO:opisać krótko zastosowane technologie, C, LKL, static variable, makefile, DejaGNU
\section{Budowa projektu}
\subsection{Diagram klas}
\subsection{Struktura plików}
\section{Szczegóły implementacji}
\subsection{Linux Kernel List}
\subsection{Zmienna kosztu? - inaczej nazwać?}
\subsection{Algortym Dijkstry}
\subsection{Testy jednostkowe - DejaGNU}
\section{Instrukcja użytkownika}
\subsection{Makefile}
\subsection{Kompilacja}
\subsection{Interfejs programisty - API}
\chapter{Porównianie z instniejącymi rozwiązaniami}
\section{Biblioteka boost::graph}
TODO:opisać co to, i jaką ma filozofie do grafów
\section{Biblioteka igraph}
\section{Testy porównawcze}

\chapter*{Podsumowanie}
\addcontentsline{toc}{chapter}{Podsumowanie}
\chapter*{Summary}
\addcontentsline{toc}{chapter}{Summary}
\bibliography{thesis}
\chapter*{Dodatek A. Dokumentacja}
\addcontentsline{toc}{chapter}{Dodatek A. Dokumentacja}
\chapter*{Dodatek B. Oświadczenie}
\addcontentsline{toc}{chapter}{Dodatek B. Oświadczenie}
\chapter*{Dodatek C. Opis zawartości płyty CD}
\addcontentsline{toc}{chapter}{Dodatek C. Opis zawartości płyty CD}

\listoffigures
\listoftables
%\printindex

\end{document}
